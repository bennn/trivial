\documentclass{article}
\usepackage{amsmath}
\usepackage{mathpartir}
\usepackage{palatino}
\begin{document}
\begin{abstract}
  The transformative \emph{and} analytic power of macros can turn
   a polymorphic type system into a dependent type system, at least
   for common programming tasks.
  By analyzing program syntax and propogating known information about program
   \emph{values} at compile-time, we can express many of the practical
   motivations for dependent types without requiring programmer annotations
   or changes to the underlying type system.

  Our macro-expanded types are not proving new theorems,
   but they recognize facts obvious to the programmer and hopefully
   give a nice experience.
\end{abstract}

Technique works by identifying common properties of a set of
 values and using these properties in a macro transformation.

\subsubsection*{Key Functions}
\begin{itemize}
\item Context-free, syntactic way of associating a Curry-style ``type'' to a value.
\subitem  $\mathsf{typeof} : Syntax \rightarrow TypeTag$
\item (Optional) Propogate metadata to future references of an identifier
\subitem $\mathsf{attach} : Syntax \times Meta \rightarrow Syntax$
\item Tag-indexed family of functions that extract data from a syntactic value.
\subitem $\mathsf{analyze} : TypeTag \rightarrow (Syntax \rightarrow Meta$
\item Transform a context based on metadata-augmented syntax for the hole.
\subitem $\mathsf{transform : (-> Context (Syntax x Meta) Syntax$
\end{itemize}

\subsubsection*{Correctness}

Suppose {\tt transform} $E~e = e'$.
A correct transformation should only catch bugs in $e$.

% Is this novel at all?
\begin{mathpar}
  \inferrule*{
    \vdash e : \tau
    \\
    \vdash e' : \tau'
  }{
    \tau' <: \tau
    \\
    e \Uparrow \iff e' \Uparrow
    \\
    e \Downarrow v \iff e' \Downarrow v
  }

  \inferrule*{
    \vdash e : \tau
    \\
    \not\vdash e' : \tau'
  }{
    e \Uparrow
  }

  \inferrule*{
    \not\vdash e : \tau
  }{
    \not\vdash e' : \tau'
  }
\end{mathpar}


\subsubsection*{Examples}

Table defines type tags and related \texttt{transformations}.

\vspace{0.4cm}
\begin{center}
\hspace{-3cm}\begin{tabular}{l l l l l}
  Type          &    Syntax Category          &    Metadata            &    Applications    \\\hline\hline
\\Format String &  {\tt string}               & \#/$\tau$ of fmt chars & Check args to {\tt printf}
\\Regexp        &  {\tt string} / {\tt bytes} & \# groups              & Check paren-matching${}^*$, smarter return type
\\Query String  &  {\tt string}               & \# of {\tt \$} vars    & Check arity
\\Function      &  {\tt (lambda ...)}         & Arity                  & Compile-time arity check for {\tt apply}
\\Numbers       &  {\tt number}               & Value                  & Constant folding, accurate types
\\Vectors       &  {\tt `\#( ... )}           & Length                 & Access, size, fast operations
\end{tabular}
\end{center}


\subsubsection*{Caveats / Extras}
\begin{itemize}
  \item Regular expressions should work with partial groups and warn about unvalidated user input.
  \subitem  i.e. \texttt{\#rx"prefix (" $+$ \#rx"mid" $+$ \#rx") suffix"} should work
  \item {Will hijack \texttt{define} to get procedure arity}
  \item {Add smarter runtime assertions, like \texttt{assert-length} and a remembering \texttt{index?}}
  \item Typeclass / Darais-class operations?
  \subitem $\circ~$ Define a {\tt TypeTag}, extend \texttt{typeof}
  \subitem $\circ~$ Write macro that looks for \& uses tags {\tt equal?}
\end{itemize}

~
\\P.S. Our goal is not to discourage dependent types, but rather to advertise macros.
\\P.P.S. Jack Firth might be using the library. Email him?
\\P.P.P.S Should work in other languages, hard part will be rename transformers


\subsubsection*{Outline}
\begin{itemize}
\item Examples of ``syntactically obvious'' dependent types.
\item Not-so-obvious examples, i.e. regular expressions \& vector operations.
\item High-level solution
\item Racket implementation, brief correctness argument
\subitem (Cool things like {\tt set!-transformer})
\item Limitations \& near misses
\item Experience report?
\end{itemize}

\end{document}
